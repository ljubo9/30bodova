\chapter{Zaključak i budući rad}
		
		17 tjedana nakon prvog okupljanja projektne grupe, naš je zadatak priveden kraju. Implementirali smo web aplikaciju za sve one koji su zainteresirani za kvalitetnu prehranu. Ostvarena je registracija i prijava korisnika u tri kategorije: "običnih" klijenata, kulinarskih entuzijasta i nutricionista. 
		"Obični" klijent na našoj KuhajIT aplikaciji ima, uz mogućnost pretraživanja kulinarskih entuzijasta i pregleda njihovih recepata, što može i neregistrirani korisnik, mogućnost skeniranja ili ručnog unosa sastojaka koje ima doma i koje želi upotrijebiti za pripravu recepta, na temelju kojih mu aplikacija generira najprikladnije recepte. Također, klijent ima mogućnost i označiti konzumirane recepte, zapratiti željene kulinarske entuzijaste te putem aplikacije pratiti dijetu koju kreira nutricionist.
		Kulinarski entuzijast ima sve mogućnosti "običnog" klijenta, no tu je i mogućnost stvaranja kuharica i recepata, koje objavljuje na aplikaciji. 
		Nutricionist također ima sve mogućnosti "običnog" klijenta, uz opciju unošenja nutritivnih podataka o sastojcima te kreiranje dijeta sa željenim ograničenjima.
		Izrada čitave ove funkcionalnosti bila je podijeljena u dva dijela: tzv. "prvi" i "drugi" ciklus.
		Prvi ciklus nije bio toliko zahtjevan što se same implementacije tiče, no ima ogromnu težinu zbog važnosti svih odluka donesenih upravo tijekom njega, od okupljanja projektne grupe, preko konceptualnog osmišljavanja same implementacije, pa sve do postavljanja čvrstih temelja na koje se naprednija implementacija nadograđivala. Značajan dio rada u prvom ciklusu bilo je opširno dokumentiranje čitavog projektnog zadatka: opis projektnog zadatka, specifikacija programske potpore (koja je uključivala obrasce uporabe te pripadajuće dijagrame obrazaca uporabe te sekvencijske dijagrame) i opis arhitekture (dizajn baze podataka te dijagrami razreda). 
		Drugi ciklus bio je prilično implementacijski zahtjevniji. Većina konkretne implementacije web aplikacije ostvarena je upravo u drugom ciklusu, uz manje, ali ne manje važne dopune dokumentaciji (dijagrami stanja, aktivnosti, komponenti, razmještaja itd.). U ovom se ciklusu većina članova tima susrela s prvim ozbiljnijim radom u njima nepoznatim tehnologijama, što je iziskivalo određeno vrijeme upoznavanja, istraživanja i prilagodbe novome. Ne treba zanemariti niti vrijeme uloženo u pažljivo testiranje rada aplikacije pomoću unit i selenium testova, a koje je opisano u jednom od gornjih poglavlja, kao niti tzv. "deployment" naše web aplikacije.
		Tijekom implementacije, dogovori između članova tima tekli su putem WhatsApp grupe, što nam se pokazalo vrlo korisnim jer su na taj način svi članovi tima, čak i oni koji u dotičnom dijelu implementacije nisu direktno sudjelovali, bili upućeni u trenutno stanje.
		Prostora za napredak uvijek ima, neke od ideja koje smo imali su stvaranje personaliziranih dijeta na zahtjev klijenata, stvaranje mobilne verzije aplikacije KuhajIT, unapređenje sučelja i dr.
		Uz sve gore navedeno, upoznavanje s novim tehnologijama, učenje pravilnog dokumentiranja, "deployanja" i slično, najvažnija tekovina ovog projekta svim je članovima upravo rad u timu. Iako nije u svim trenucima bilo jednostavno, svi smo naučili vrijedne lekcije o timskom radu, međusobnoj potpori, organizaciji vremena i vrijednostima koje ćemo tražiti i u nekim budućim projektnim grupama.
		S obzirom na to da nam je svima to prvo iskustvo ovakvoga tipa, a i na ostale obaveze koje svi imamo na fakultetu, ponosni smo na obavljeno.
				\eject 